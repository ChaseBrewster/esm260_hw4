\documentclass[]{article}
\usepackage{lmodern}
\usepackage{amssymb,amsmath}
\usepackage{ifxetex,ifluatex}
\usepackage{fixltx2e} % provides \textsubscript
\ifnum 0\ifxetex 1\fi\ifluatex 1\fi=0 % if pdftex
  \usepackage[T1]{fontenc}
  \usepackage[utf8]{inputenc}
\else % if luatex or xelatex
  \ifxetex
    \usepackage{mathspec}
  \else
    \usepackage{fontspec}
  \fi
  \defaultfontfeatures{Ligatures=TeX,Scale=MatchLowercase}
\fi
% use upquote if available, for straight quotes in verbatim environments
\IfFileExists{upquote.sty}{\usepackage{upquote}}{}
% use microtype if available
\IfFileExists{microtype.sty}{%
\usepackage{microtype}
\UseMicrotypeSet[protrusion]{basicmath} % disable protrusion for tt fonts
}{}
\usepackage[margin=1in]{geometry}
\usepackage{hyperref}
\hypersetup{unicode=true,
            pdftitle={ESM 260: Applied Marine Ecology},
            pdfauthor={Chase Brewster},
            pdfborder={0 0 0},
            breaklinks=true}
\urlstyle{same}  % don't use monospace font for urls
\usepackage{color}
\usepackage{fancyvrb}
\newcommand{\VerbBar}{|}
\newcommand{\VERB}{\Verb[commandchars=\\\{\}]}
\DefineVerbatimEnvironment{Highlighting}{Verbatim}{commandchars=\\\{\}}
% Add ',fontsize=\small' for more characters per line
\usepackage{framed}
\definecolor{shadecolor}{RGB}{248,248,248}
\newenvironment{Shaded}{\begin{snugshade}}{\end{snugshade}}
\newcommand{\AlertTok}[1]{\textcolor[rgb]{0.94,0.16,0.16}{#1}}
\newcommand{\AnnotationTok}[1]{\textcolor[rgb]{0.56,0.35,0.01}{\textbf{\textit{#1}}}}
\newcommand{\AttributeTok}[1]{\textcolor[rgb]{0.77,0.63,0.00}{#1}}
\newcommand{\BaseNTok}[1]{\textcolor[rgb]{0.00,0.00,0.81}{#1}}
\newcommand{\BuiltInTok}[1]{#1}
\newcommand{\CharTok}[1]{\textcolor[rgb]{0.31,0.60,0.02}{#1}}
\newcommand{\CommentTok}[1]{\textcolor[rgb]{0.56,0.35,0.01}{\textit{#1}}}
\newcommand{\CommentVarTok}[1]{\textcolor[rgb]{0.56,0.35,0.01}{\textbf{\textit{#1}}}}
\newcommand{\ConstantTok}[1]{\textcolor[rgb]{0.00,0.00,0.00}{#1}}
\newcommand{\ControlFlowTok}[1]{\textcolor[rgb]{0.13,0.29,0.53}{\textbf{#1}}}
\newcommand{\DataTypeTok}[1]{\textcolor[rgb]{0.13,0.29,0.53}{#1}}
\newcommand{\DecValTok}[1]{\textcolor[rgb]{0.00,0.00,0.81}{#1}}
\newcommand{\DocumentationTok}[1]{\textcolor[rgb]{0.56,0.35,0.01}{\textbf{\textit{#1}}}}
\newcommand{\ErrorTok}[1]{\textcolor[rgb]{0.64,0.00,0.00}{\textbf{#1}}}
\newcommand{\ExtensionTok}[1]{#1}
\newcommand{\FloatTok}[1]{\textcolor[rgb]{0.00,0.00,0.81}{#1}}
\newcommand{\FunctionTok}[1]{\textcolor[rgb]{0.00,0.00,0.00}{#1}}
\newcommand{\ImportTok}[1]{#1}
\newcommand{\InformationTok}[1]{\textcolor[rgb]{0.56,0.35,0.01}{\textbf{\textit{#1}}}}
\newcommand{\KeywordTok}[1]{\textcolor[rgb]{0.13,0.29,0.53}{\textbf{#1}}}
\newcommand{\NormalTok}[1]{#1}
\newcommand{\OperatorTok}[1]{\textcolor[rgb]{0.81,0.36,0.00}{\textbf{#1}}}
\newcommand{\OtherTok}[1]{\textcolor[rgb]{0.56,0.35,0.01}{#1}}
\newcommand{\PreprocessorTok}[1]{\textcolor[rgb]{0.56,0.35,0.01}{\textit{#1}}}
\newcommand{\RegionMarkerTok}[1]{#1}
\newcommand{\SpecialCharTok}[1]{\textcolor[rgb]{0.00,0.00,0.00}{#1}}
\newcommand{\SpecialStringTok}[1]{\textcolor[rgb]{0.31,0.60,0.02}{#1}}
\newcommand{\StringTok}[1]{\textcolor[rgb]{0.31,0.60,0.02}{#1}}
\newcommand{\VariableTok}[1]{\textcolor[rgb]{0.00,0.00,0.00}{#1}}
\newcommand{\VerbatimStringTok}[1]{\textcolor[rgb]{0.31,0.60,0.02}{#1}}
\newcommand{\WarningTok}[1]{\textcolor[rgb]{0.56,0.35,0.01}{\textbf{\textit{#1}}}}
\usepackage{graphicx,grffile}
\makeatletter
\def\maxwidth{\ifdim\Gin@nat@width>\linewidth\linewidth\else\Gin@nat@width\fi}
\def\maxheight{\ifdim\Gin@nat@height>\textheight\textheight\else\Gin@nat@height\fi}
\makeatother
% Scale images if necessary, so that they will not overflow the page
% margins by default, and it is still possible to overwrite the defaults
% using explicit options in \includegraphics[width, height, ...]{}
\setkeys{Gin}{width=\maxwidth,height=\maxheight,keepaspectratio}
\IfFileExists{parskip.sty}{%
\usepackage{parskip}
}{% else
\setlength{\parindent}{0pt}
\setlength{\parskip}{6pt plus 2pt minus 1pt}
}
\setlength{\emergencystretch}{3em}  % prevent overfull lines
\providecommand{\tightlist}{%
  \setlength{\itemsep}{0pt}\setlength{\parskip}{0pt}}
\setcounter{secnumdepth}{0}
% Redefines (sub)paragraphs to behave more like sections
\ifx\paragraph\undefined\else
\let\oldparagraph\paragraph
\renewcommand{\paragraph}[1]{\oldparagraph{#1}\mbox{}}
\fi
\ifx\subparagraph\undefined\else
\let\oldsubparagraph\subparagraph
\renewcommand{\subparagraph}[1]{\oldsubparagraph{#1}\mbox{}}
\fi

%%% Use protect on footnotes to avoid problems with footnotes in titles
\let\rmarkdownfootnote\footnote%
\def\footnote{\protect\rmarkdownfootnote}

%%% Change title format to be more compact
\usepackage{titling}

% Create subtitle command for use in maketitle
\newcommand{\subtitle}[1]{
  \posttitle{
    \begin{center}\large#1\end{center}
    }
}

\setlength{\droptitle}{-2em}

  \title{ESM 260: Applied Marine Ecology}
    \pretitle{\vspace{\droptitle}\centering\huge}
  \posttitle{\par}
  \subtitle{Problem Set 4}
  \author{Chase Brewster}
    \preauthor{\centering\large\emph}
  \postauthor{\par}
      \predate{\centering\large\emph}
  \postdate{\par}
    \date{3/03/2020}

\usepackage{setspace}

\doublespacing
\usepackage{fontspec}
\setmainfont{Calibri Light}

\begin{document}
\maketitle

\#Load Packages \& Read in Data

\begin{Shaded}
\begin{Highlighting}[]
\KeywordTok{library}\NormalTok{(tidyverse)}
\KeywordTok{library}\NormalTok{(wesanderson)}

\NormalTok{climate <-}\StringTok{ }\KeywordTok{read_csv}\NormalTok{(}\StringTok{"climate.csv"}\NormalTok{)}
\NormalTok{crab <-}\StringTok{ }\KeywordTok{read_csv}\NormalTok{(}\StringTok{"crab.csv"}\NormalTok{)}
\NormalTok{nuclear <-}\StringTok{ }\KeywordTok{read_csv}\NormalTok{(}\StringTok{"nuclear.csv"}\NormalTok{)}
\NormalTok{upwell <-}\StringTok{ }\KeywordTok{read_csv}\NormalTok{(}\StringTok{"upwell.csv"}\NormalTok{)}
\end{Highlighting}
\end{Shaded}

\begin{center}\rule{0.5\linewidth}{\linethickness}\end{center}

\#1. Climate Change

First, there is growing pressure in Congress to relax environmental
regulations affecting carbon dioxide emissions. You have been asked to
provide evidence for changes in marine communities that may be due to
climatic changes, such as increases in temperature. Your have data from
a coastal site in Northern California that had been previously surveyed
in 1955. During the ensuing 65 years there have been substantial changes
in the abundances of several species. To help interpret the changes, you
can look at the following information for all species: 1) the mode of
development (i.e., planktonic larvae vs.~non-planktonic larvae), 2) the
geographic range (i.e., does the species range occur primarily North or
South of the site, or is it Cosmopolitan - both north and south), and 3)
the trophic level (i.e., predator, filter feeder, scavenger). What are
the patterns of change? Try grouping the species by categories and see
if anything emerges.

\begin{Shaded}
\begin{Highlighting}[]
\CommentTok{##going to start by filtering for different qualities of the species and see if anything emerges}

\NormalTok{climate_explore <-}\StringTok{ }\NormalTok{climate }\OperatorTok\StringTok{ }
\StringTok{  }\KeywordTok{filter}\NormalTok{(trophic }\OperatorTok{==}\StringTok{ "scavenger"}\NormalTok{)}

\CommentTok{##going to run a linear model to see if any species category is significant}

\NormalTok{climate_lm <-}\StringTok{ }\KeywordTok{lm}\NormalTok{(density }\OperatorTok{~}\StringTok{ }\NormalTok{development}\OperatorTok{+}\NormalTok{range}\OperatorTok{+}\NormalTok{trophic, }\DataTypeTok{data =}\NormalTok{ climate)}

\KeywordTok{summary}\NormalTok{(climate_lm)}
\end{Highlighting}
\end{Shaded}

\begin{verbatim}
## 
## Call:
## lm(formula = density ~ development + range + trophic, data = climate)
## 
## Residuals:
##    Min     1Q Median     3Q    Max 
## -48.52 -17.65   0.08  13.74  57.41 
## 
## Coefficients:
##                       Estimate Std. Error t value Pr(>|t|)    
## (Intercept)             12.611     14.973   0.842 0.413791    
## developmentplanktonic   20.268     15.523   1.306 0.212713    
## rangeNorthern          -91.284     19.617  -4.653 0.000373 ***
## rangeSouthern           56.446     18.338   3.078 0.008181 ** 
## trophicpredator         -2.011     17.221  -0.117 0.908714    
## trophicscavenger       -23.802     18.378  -1.295 0.216234    
## ---
## Signif. codes:  0 '***' 0.001 '**' 0.01 '*' 0.05 '.' 0.1 ' ' 1
## 
## Residual standard error: 31.61 on 14 degrees of freedom
## Multiple R-squared:  0.8093, Adjusted R-squared:  0.7412 
## F-statistic: 11.88 on 5 and 14 DF,  p-value: 0.0001241
\end{verbatim}

\begin{itemize}
\item
  there does not seem to be any patterns in planktonic
  vs.~non-planktonic
\item
  Northern species have all decreased in density (significant)
\item
  Southern species have all increased in density (significant)
\item
  there is no pattern for cosmopolitan
\item
  there does not seem to be a pattern for predators
\item
  there does not seem to be a pattern for filter feeders
\item
  there does not seem to be a pattern for scavengers (all but 1
  increase, but they are all southern \& cosmo)
\end{itemize}

\begin{Shaded}
\begin{Highlighting}[]
\CommentTok{##Now going to make graphs for each species category}

\NormalTok{climate_graph_range <-}\StringTok{ }\KeywordTok{ggplot}\NormalTok{(climate, }\KeywordTok{aes}\NormalTok{(}\DataTypeTok{x =}\NormalTok{ species, }\DataTypeTok{y =}\NormalTok{ density)) }\OperatorTok{+}
\StringTok{  }\KeywordTok{geom_col}\NormalTok{() }\OperatorTok{+}\StringTok{ }
\StringTok{  }\KeywordTok{facet_wrap}\NormalTok{(}\OperatorTok{~}\NormalTok{range) }\OperatorTok{+}
\StringTok{  }\KeywordTok{theme_light}\NormalTok{()}

\NormalTok{climate_graph_range}
\end{Highlighting}
\end{Shaded}

\includegraphics{hw4_files/figure-latex/climate change graph-1.pdf}

\begin{Shaded}
\begin{Highlighting}[]
\NormalTok{climate_graph_development <-}\StringTok{ }\KeywordTok{ggplot}\NormalTok{(climate, }\KeywordTok{aes}\NormalTok{(}\DataTypeTok{x =}\NormalTok{ species, }\DataTypeTok{y =}\NormalTok{ density)) }\OperatorTok{+}
\StringTok{  }\KeywordTok{geom_col}\NormalTok{() }\OperatorTok{+}\StringTok{ }
\StringTok{  }\KeywordTok{facet_wrap}\NormalTok{(}\OperatorTok{~}\NormalTok{development) }\OperatorTok{+}
\StringTok{  }\KeywordTok{theme_light}\NormalTok{()}

\NormalTok{climate_graph_development}
\end{Highlighting}
\end{Shaded}

\includegraphics{hw4_files/figure-latex/climate change graph-2.pdf}

\begin{Shaded}
\begin{Highlighting}[]
\NormalTok{climate_graph_trophic <-}\StringTok{ }\KeywordTok{ggplot}\NormalTok{(climate, }\KeywordTok{aes}\NormalTok{(}\DataTypeTok{x =}\NormalTok{ species, }\DataTypeTok{y =}\NormalTok{ density)) }\OperatorTok{+}
\StringTok{  }\KeywordTok{geom_col}\NormalTok{() }\OperatorTok{+}\StringTok{ }
\StringTok{  }\KeywordTok{facet_wrap}\NormalTok{(}\OperatorTok{~}\NormalTok{trophic) }\OperatorTok{+}
\StringTok{  }\KeywordTok{theme_light}\NormalTok{()}

\NormalTok{climate_graph_trophic}
\end{Highlighting}
\end{Shaded}

\includegraphics{hw4_files/figure-latex/climate change graph-3.pdf} ***

\#2. Nuclear Power Plant

Second, El Diablo nuclear power plant is up for a renewal of its
operating permit. One issue of concern is whether the plant has been
altering the marine community near its discharge point (Diablo Cove). A
monitoring program, which started ten years before the plant became
operational in 2010, followed the abundance of 3 species of
invertebrates at the discharge point and at two sites (Hyperion Bay \&
Noway Bay) that appeared to be at least superficially similar to the
discharge point. You now also have ten years of monitoring data from the
sites after the plant became operational. Are there any detectable
impacts of the plant on these species? Which site(s) did you use as a
control? Why?

\begin{Shaded}
\begin{Highlighting}[]
\CommentTok{#diablo is the operating location}
\CommentTok{#became operational in 2010}
\CommentTok{#3 species}

\CommentTok{#create datasets for each species}

\NormalTok{nuclear_a <-}\StringTok{ }\NormalTok{nuclear }\OperatorTok\StringTok{ }
\StringTok{  }\KeywordTok{filter}\NormalTok{(species }\OperatorTok{==}\StringTok{ "A"}\NormalTok{) }

\NormalTok{nuclear_b <-}\StringTok{ }\NormalTok{nuclear }\OperatorTok\StringTok{ }
\StringTok{  }\KeywordTok{filter}\NormalTok{(species }\OperatorTok{==}\StringTok{ "B"}\NormalTok{)}

\NormalTok{nuclear_c <-}\StringTok{ }\NormalTok{nuclear }\OperatorTok\StringTok{ }
\StringTok{  }\KeywordTok{filter}\NormalTok{(species }\OperatorTok{==}\StringTok{ "C"}\NormalTok{)}
\end{Highlighting}
\end{Shaded}

\begin{Shaded}
\begin{Highlighting}[]
\CommentTok{#Species A}
\NormalTok{nuclear_a <-}\StringTok{ }\NormalTok{nuclear_a }\OperatorTok\StringTok{ }
\StringTok{  }\KeywordTok{group_by}\NormalTok{(location)}

\NormalTok{nuc_a_graph <-}\StringTok{ }\KeywordTok{ggplot}\NormalTok{(nuclear_a, }\KeywordTok{aes}\NormalTok{(}\DataTypeTok{x =}\NormalTok{ year, }\DataTypeTok{y =}\NormalTok{ density)) }\OperatorTok{+}
\StringTok{  }\KeywordTok{geom_line}\NormalTok{(}\KeywordTok{aes}\NormalTok{(}\DataTypeTok{color =}\NormalTok{ location), }\DataTypeTok{size =} \DecValTok{1}\NormalTok{) }\OperatorTok{+}
\StringTok{  }\KeywordTok{theme_light}\NormalTok{() }\OperatorTok{+}
\StringTok{  }\KeywordTok{scale_color_manual}\NormalTok{(}\DataTypeTok{values=}\KeywordTok{wes_palette}\NormalTok{(}\DataTypeTok{n=}\DecValTok{3}\NormalTok{, }\DataTypeTok{name=}\StringTok{"FantasticFox1"}\NormalTok{)) }\OperatorTok{+}
\StringTok{  }\KeywordTok{geom_vline}\NormalTok{(}\DataTypeTok{xintercept =} \DecValTok{2010}\NormalTok{) }\OperatorTok{+}
\StringTok{  }\KeywordTok{geom_smooth}\NormalTok{(}\DataTypeTok{method =} \StringTok{"lm"}\NormalTok{, }\DataTypeTok{formula =}\NormalTok{ y }\OperatorTok{~}\StringTok{ }\NormalTok{x, }\KeywordTok{aes}\NormalTok{(}\DataTypeTok{color =}\NormalTok{ location), }\DataTypeTok{se =} \OtherTok{FALSE}\NormalTok{, }\DataTypeTok{size =} \FloatTok{0.8}\NormalTok{)}

\NormalTok{nuc_a_graph}
\end{Highlighting}
\end{Shaded}

\includegraphics{hw4_files/figure-latex/nuclear power graph-1.pdf}

\begin{Shaded}
\begin{Highlighting}[]
\CommentTok{#Species B}
\NormalTok{nuclear_b <-}\StringTok{ }\NormalTok{nuclear_b }\OperatorTok\StringTok{ }
\StringTok{  }\KeywordTok{group_by}\NormalTok{(location)}

\NormalTok{nuc_b_graph <-}\StringTok{ }\KeywordTok{ggplot}\NormalTok{(nuclear_b, }\KeywordTok{aes}\NormalTok{(}\DataTypeTok{x =}\NormalTok{ year, }\DataTypeTok{y =}\NormalTok{ density)) }\OperatorTok{+}
\StringTok{  }\KeywordTok{geom_line}\NormalTok{(}\KeywordTok{aes}\NormalTok{(}\DataTypeTok{color =}\NormalTok{ location), }\DataTypeTok{size =} \DecValTok{1}\NormalTok{) }\OperatorTok{+}
\StringTok{  }\KeywordTok{theme_light}\NormalTok{() }\OperatorTok{+}
\StringTok{  }\KeywordTok{scale_color_manual}\NormalTok{(}\DataTypeTok{values=}\KeywordTok{wes_palette}\NormalTok{(}\DataTypeTok{n=}\DecValTok{3}\NormalTok{, }\DataTypeTok{name=}\StringTok{"FantasticFox1"}\NormalTok{)) }\OperatorTok{+}
\StringTok{  }\KeywordTok{geom_vline}\NormalTok{(}\DataTypeTok{xintercept =} \DecValTok{2010}\NormalTok{) }\OperatorTok{+}
\StringTok{  }\KeywordTok{geom_smooth}\NormalTok{(}\DataTypeTok{method =} \StringTok{"lm"}\NormalTok{, }\DataTypeTok{formula =}\NormalTok{ y }\OperatorTok{~}\StringTok{ }\NormalTok{x, }\KeywordTok{aes}\NormalTok{(}\DataTypeTok{color =}\NormalTok{ location), }\DataTypeTok{se =} \OtherTok{FALSE}\NormalTok{, }\DataTypeTok{size =} \FloatTok{0.8}\NormalTok{)}

\NormalTok{nuc_b_graph}
\end{Highlighting}
\end{Shaded}

\includegraphics{hw4_files/figure-latex/nuclear power graph-2.pdf}

\begin{Shaded}
\begin{Highlighting}[]
\CommentTok{#Species C}
\NormalTok{nuclear_c <-}\StringTok{ }\NormalTok{nuclear_c }\OperatorTok\StringTok{ }
\StringTok{  }\KeywordTok{group_by}\NormalTok{(location)}

\NormalTok{nuc_c_graph <-}\StringTok{ }\KeywordTok{ggplot}\NormalTok{(nuclear_c, }\KeywordTok{aes}\NormalTok{(}\DataTypeTok{x =}\NormalTok{ year, }\DataTypeTok{y =}\NormalTok{ density)) }\OperatorTok{+}
\StringTok{  }\KeywordTok{geom_line}\NormalTok{(}\KeywordTok{aes}\NormalTok{(}\DataTypeTok{color =}\NormalTok{ location), }\DataTypeTok{size =} \DecValTok{1}\NormalTok{) }\OperatorTok{+}
\StringTok{  }\KeywordTok{theme_light}\NormalTok{() }\OperatorTok{+}
\StringTok{  }\KeywordTok{scale_color_manual}\NormalTok{(}\DataTypeTok{values=}\KeywordTok{wes_palette}\NormalTok{(}\DataTypeTok{n=}\DecValTok{3}\NormalTok{, }\DataTypeTok{name=}\StringTok{"FantasticFox1"}\NormalTok{)) }\OperatorTok{+}
\StringTok{  }\KeywordTok{geom_vline}\NormalTok{(}\DataTypeTok{xintercept =} \DecValTok{2010}\NormalTok{) }\OperatorTok{+}
\StringTok{  }\KeywordTok{geom_smooth}\NormalTok{(}\DataTypeTok{method =} \StringTok{"lm"}\NormalTok{, }\DataTypeTok{formula =}\NormalTok{ y }\OperatorTok{~}\StringTok{ }\NormalTok{x, }\KeywordTok{aes}\NormalTok{(}\DataTypeTok{color =}\NormalTok{ location), }\DataTypeTok{se =} \OtherTok{FALSE}\NormalTok{, }\DataTypeTok{size =} \FloatTok{0.8}\NormalTok{)}

\NormalTok{nuc_c_graph}
\end{Highlighting}
\end{Shaded}

\includegraphics{hw4_files/figure-latex/nuclear power graph-3.pdf}

\begin{itemize}
\item
  there is no discernable trend for pre-2010 and post-2010 for Species A
\item
  there is no discernable trend for pre-2010 and post-2010 for Species B
\item
  there appears be a slightly greater increase for Species C post-2010
  relative to the control
\item
  Hyperion should serve as the control, relative to noway, as noway is
  highly variable and hyperion densities are closely aligned to diablo
  densities pre-2010
\end{itemize}

\begin{center}\rule{0.5\linewidth}{\linethickness}\end{center}

\#3. Fisheries and Invasive Species

Finally, you have been asked to make recommendations regarding 2 key
species of interest. One species is a crab (Cancer cancer). There is a
great deal of interest in developing fisheries for Cancer cancer in both
Oregon and California. The other species is a barnacle (Barnacle bill).
Barnacle bill invaded the US West Coast possibly from Japan earlier in
the century. It has become a nuisance species and regularly fouls pipes,
ships etc. There is a great deal of interest in learning more about the
dynamics of this species in an effort to eliminate it.


\end{document}
